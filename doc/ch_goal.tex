
\section{Sformułowanie zadania.}
\emph{Wstępująca i zstępująca dyskretyzacja atrybutów ciągłych z uwzględnieniem rozkładu kategorii. Badanie wpływu dyskretyzacji na jakość modeli klasyfikacji tworzonych za pomocą algorytmów dostępnych w R.}

\section{Rozwinięcie tematu.}
Celem projektu była implementacj dyskretyzacji z nadzorem, globalnej. Zostały zaimplementowane dwie metody: zstępująca i wstępująca, z których każda jest parametryzowalna poprzez podawanie odpowiednich kryteriów stopu (bądź ich kombinacji), wraz z ewentualnymi parametrami. Na potrzeby porównania jakości dyskretyzacji uzyskanej metodą \emph{BottomUp} i \emph{TopDown} konieczne było uzyskanie tej samej liczby przedziałów dla obu metod. W tym celu zaimplementowano dodatkowe kryterium stopu: żądaną ilość przedziałów. W związku z tym dla obu metod zaimplementowano globalne kryteria wyboru przedziałów do łączenia/podziału (ważenie wskaźników jakości przedziału licznością przedziału) - tak aby w danym momencie działania algorytmu, algorytm wybierał globalnie najlepsze przedziały do łączenia/podziału.
