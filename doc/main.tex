\documentclass[11pt,a4paper]{report}
\usepackage{polski}
\usepackage[utf8]{inputenc}
\usepackage[colorlinks=true,linkcolor=black,urlcolor=blue,citecolor=RoyalBlue]{hyperref}
\usepackage[usenames,dvipsnames]{color}
\usepackage{alltt}
\usepackage{booktabs} % eleganckie tabelki
\usepackage{graphicx}

\newcounter{liczp}
\newenvironment{example}{\refstepcounter{liczp}{\noindent
{\bf Przykład~\theliczp:}\,}}


\addtolength{\textwidth}{4cm}
\addtolength{\hoffset}{-2cm}
\addtolength{\textheight}{4cm}
\addtolength{\voffset}{-2cm}
\date {\today}
\author {Aleksy Barcz, Wojciech Koszołko}
\title{Dyskretyzacja atrybutów}
\begin{document}
\maketitle
\tableofcontents

\bibliographystyle{plain}

\chapter{Cel projektu}

\section{Sformułowanie zadania.}
\emph{Wstępująca i zstępująca dyskretyzacja atrybutów ciągłych z uwzględnieniem rozkładu kategorii. Badanie wpływu dyskretyzacji na jakość modeli klasyfikacji tworzonych za pomocą algorytmów dostępnych w R.}

\section{Rozwinięcie tematu.}
Celem projektu była implementacj dyskretyzacji z nadzorem, globalnej. Każdy atrybut był dyskretyzowany niezależnie od pozostałych. Zostały zaimplementowane dwie metody: zstępująca i wstępująca, z których każda jest parametryzowalna poprzez podawanie odpowiednich kryteriów stopu (bądź ich kombinacji), wraz z ewentualnymi parametrami. Na potrzeby porównania jakości dyskretyzacji uzyskanej metodą \emph{BottomUp} i \emph{TopDown} konieczne było uzyskanie tej samej liczby przedziałów dla obu metod. W tym celu zaimplementowano dodatkowe kryterium stopu: żądaną ilość przedziałów. W związku z tym dla obu metod zaimplementowano globalne kryteria wyboru przedziałów do łączenia/podziału (ważenie wskaźników jakości przedziału licznością przedziału) - tak aby w danym momencie działania algorytmu, algorytm wybierał globalnie najlepsze przedziały do łączenia/podziału.


\chapter{Dyskretyzacja zstępująca}

Jako kryterium podziału dla dyskretyzacji zstępującej (\emph{TopDown}) zaimplementowano wybór progu maksymalizującego spadek entropii dla wybranego przedziału w przypadku podzielenia przedziału na dwa nowe przedziały względem tego progu. Schemat działania kryterium podziału:
\begin{enumerate}
	\item{Dla każdego przedziału:}
	\begin{enumerate}
		\item{obliczenie progu $\theta$ maksymalizującego spadek entropii.}
		\item{obliczenie ważonego (ilością próbek) spadku entropii dla wybranego progu.}
	\end{enumerate}
	\item{Wybranie przedziału o maksymalnym potencjalnym ważonym spadku entropii.}
	\item{Podział wybranego przedziału na dwa nowe, względem wyznaczonego progu.}
\end{enumerate}
Zaimplementowane kryteria stopu:
\begin{itemize}
	\item{żądana ilość przedziałów -- algorytm stara się osiągnąć zadaną ilość przedziałów, nawet gdy kolejne podziały nie dają spadku entropii; uzyskana ilość przedziałów może być mniejsza od zadanej, gdy żadnego przedziału nie da się już podzielić (takie same wartości atrybutu $a$ w ramach każdego przedziału lub wszystkie przedziały jednoelementowe)}
	\item{minimalny potencjalny spadek entropii (warunek globalny) -- zatrzymanie algorytmu następuje gdy dla żadnego z przedziałów nie da się osiągnąć zadanego jako parametr spadku entropii (nieważonego)}
	\item{kryterium \emph{delta}~(\ref{eq:delta_criterion}), łączące jakość kodowania z entropią~\cite{cichosz2000systemy}}
\end{itemize}

\begin{equation}
\displaystyle g_{a,\theta}(P) < \frac{\log(\vert P\vert-1)}{\vert P\vert}+ \frac{\Delta_{a,\theta}(P)}{\vert P\vert},
\label{eq:delta_criterion}
\end{equation}

\begin{equation}
\displaystyle \Delta_{a,\theta}(P) = \log(3^{\vert C_P\vert}-2) - \big(\vert C_P\vert I(P) - \vert C_{P_{a\leq\theta}}\vert E_{a\leq\theta}(P) - \vert C_{P_{a>\theta}}\vert E_{a>\theta}(P)\big),
\end{equation}

\begin{equation}
\displaystyle C_P = \{d\in C \;\vert\; (\exists x\in P)\;c(x)=d\}.
\end{equation}


\chapter{Dyskretyzacja wstępująca}
Jako kryterium złączenia dwóch przyległych przedziałów dla dyskretyzacji wstępującej BottomUp zaimplementowano analizę statystki $\chi^2$. Na podstawie wartości aktualnie rozpatrywanego atrybutu, tworzone są przedziały (każda unikalna wartość w oddzielnym przedziale). Następnie przedziały są łączone aż osiągną kryterium stopu. Schemat działania kryterium połączenia dwóch przyległy przedziałów:
\begin{enumerate}
	\item{Dla każdego przyległego przedziału:}
	\begin{enumerate}
		\item{Oblicz wartość $\chi^2$}
	\end{enumerate}
	\item{Połącz dwa przedziały o minimalnej wartości $\chi^2$}
\end{enumerate}
Wartość statyski $\chi^2$ pomiędzy dwoma przedziałami jest liczona na podstawie wzoru: \\

\begin{center}
$\chi^2 = \sum_{i=1}^{2}\sum_{j}^{c}\frac{(A_{ij}-E_{ij})^2}{E_{ij}}$
\end{center}
Gdzie:
\\$c$ = liczba k
\\$A_{ij}$ = liczba wartości w i-tym przedziale, j-tej klasy
\\$R_{i}$ = liczba wartości w i-tym przedziale
\\$C_{j}$ = liczba obiektów j-tej klasy w obu przedziałach
\\$N$ = liczba wartości w obu przedziałach
\\$E_{ij} = (R+{i}*C_{j})/N$
\\Zaimplementowane kryteria stopu:
\begin{itemize}
	\item{żądana ilość przedziałów -- algorytm stara się osiągnąć zadaną ilość przedziałów; uzyskana ilość przedziałów może być mniejsza od zadanej, gdy zostanie dodane kryterium  na maksymalną wartość $\chi^2$, dla której następuje złączenie przyległych przedziałów. }
	\item{maksymalna wartość $\chi^2$) -- zatrzymanie algorytmu następuje gdy dla żadnej z pary przedziałów nie dało się osiągnąć wartości $\chi^2$ mniejsze od zadanej w kryterium}
\end{itemize}


\chapter{Testy}
\section{Specyfikacja ogólna testów.}
 Testy zostały wykonane przy użyciu dwóch zbiorów danych - uczącego i testowego, zgodnie z następującym schematem:
\begin{itemize}
	\item{Zbudowanie modelu dyskretyzacji na zbiorze uczącym}
	\item{Dyskretyzacja zbioru uczącego przy użyciu modelu dyskretyzacji}
	\item{Nauczenie klasyfikatora na zdyskretyzowanym zbiorze uczącym}
	\item{Dyskretyzacja zbioru testowego przy użyciu modelu dyskretyzacji}
	\item{Sprawdzenie jakości klasyfikacji na zdyskretyzowanym zbiorze uczącym}
\end{itemize}
Dla uzyskania wiarygodnych wyników zastosowano we wszystkich eksperymentach walidację krzyżową. Do testów użyto zbiór danych pomiarowych opisujących procesy zachodzące na czujnikach chemicznych \cite{Gas:2012}. Zbiór danych składa się z próbek o 129 atrybutach o wartościach ciągłych, na potrzeby eksperymentów dyskretyzowano wszystkie bądź wybrane atrybuty. Ze względu na długi czas dyskretyzacji do eksperymentów wykorzystano tylko zbiór \emph{batch1.dat}, składający się ze 445 próbek o dość równomiernym rozkładzie 6ciu klas. W celu zmniejszenia czasu eksperymentów, część z nich przeprowadzono dyskretyzując tylko atrybuty najistotniejsze z punktu widzenia klasy próbek. Zbiór wybranych atrybutów to: $\{V2, V7, V11, V18, V20, V26, V51, V67\}$. Atrybuty te zostały wybrane przy użyciu drzewa decyzyjnego z pakietu \emph{rpart}.

\newpage
\begin{alltt}
rpart(V1 ~ ., gas1)
n= 445 

node), split, n, loss, yval, (yprob)
      * denotes terminal node

 1) root 445 347 2 (0.2 0.22 0.19 0.067 0.16 0.17)  
   2) V2< 31988.61 189  91 2 (0.048 0.52 0.43 0 0 0)  
     4) V51>=5.942954 89   2 2 (0 0.98 0.022 0 0 0) *
     5) V51< 5.942954 100  20 3 (0.09 0.11 0.8 0 0 0)  
      10) V26< 8863.391 21  10 2 (0.43 0.52 0.048 0 0 0)  
        20) V7< -1.138504 10   1 1 (0.9 0 0.1 0 0 0) *
        21) V7>=-1.138504 11   0 2 (0 1 0 0 0 0) *
      11) V26>=8863.391 79   0 3 (0 0 1 0 0 0) *
   3) V2>=31988.61 256 175 1 (0.32 0 0.0039 0.12 0.27 0.29)  
     6) V2< 194291.8 177  96 1 (0.46 0 0.0056 0.062 0.056 0.42)  
      12) V18>=7439.422 71   3 1 (0.96 0 0.014 0.028 0 0) *
      13) V18< 7439.422 106  32 6 (0.12 0 0 0.085 0.094 0.7)  
        26) V67< 5.20329 29  16 1 (0.45 0 0 0.1 0.34 0.1)  
          52) V20>=0.8470795 13   0 1 (1 0 0 0 0 0) *
          53) V20< 0.8470795 16   6 5 (0 0 0 0.19 0.62 0.19) *
        27) V67>=5.20329 77   6 6 (0 0 0 0.078 0 0.92) *
     7) V2>=194291.8 79  19 5 (0 0 0 0.24 0.76 0)  
      14) V11>=9.623032 21   2 4 (0 0 0 0.9 0.095 0) *
      15) V11< 9.623032 58   0 5 (0 0 0 0 1 0) *
\end{alltt}



\section{Naiwny klasyfikator Bayesa.}
Do testów wykorzystano naiwny klasyfikator Bayesa z pakietu \emph{e1071}. Klasyfikator Bayesa dla atrybutów o wartościach dyskretnych liczy bezpośrednio prawdopodobieństwa warunkowe, natomiast w przypadku atrybutów o wartościach ciągłych estymuje dla każdego atrybutu rozkład prawdopodobieństwa, co wiąże się z pewną niedokładnością predykcji. W związku z tym dobrze przeprowadzona dyskretyzacja atrybutu powinna polepszyć jakość klasyfikacji, dzięki uniknięciu estymacji rozkładu prawdopodobieństwa wartości atrybutu. W ramach eksperymentu przeprowadzono 5-krotną walidację krzyżową na zbiorze danych, wyniki zestawiono w tabeli~\ref{tab:bayes_full_set}. Wyniki 5-krotnej walidacji krzyżowej na zbiorze danych z ograniczonym zbiorem dyskretyzowanych atrybutów zestawiono w tabeli~\ref{tab:bayes_reduced_set}.

\begin{table}[h!]
\begin{center}
\begin{tabular}{lrrrr}
\toprule
metoda & kryterium stopu & parametr & średnia dokładność & odchylenie std \\
\midrule
--       & --    & -- & 0.7592 & 0.0306 \\
TopDown  & delta & -- & 0.8498 & 0.0436 \\
\bottomrule
\end{tabular}
\caption{Dokładność klasyfikacji - klasyfikator Bayesa, 5-krotna walidacja krzyżowa}
\label{tab:bayes_full_set}
\end{center}

\end{table}
\begin{table}[h!]
\begin{center}
\begin{tabular}{lrrrr}
\toprule
metoda & kryterium stopu & parametr & średnia dokładność & odchylenie std \\
\midrule
--       & --    & -- & 0.8711 & 0.0435 \\
TopDown  & delta & -- & 0.8835 & 0.0158 \\
\bottomrule
\end{tabular}
\caption{Dokładność klasyfikacji - klasyfikator Bayesa, 5-krotna walidacja krzyżowa, ograniczony zbiór atrybutów dyskretyzowanych}
\label{tab:bayes_reduced_set}
\end{center}
\end{table}

\subsection{Porównanie działania metody wstępującej i zstępującej.}
Jakość dyskretyzacji wstępującej i zstępującej porównano dla zadanej liczby przedziałów dla każdego dyskretyzowanego atrybutu. W tabeli~\ref{tab:bayes_td_bu_comp_full_set} zestawiono jakość klasyfikacji przy użyciu klasyfikatora Bayesa.

\begin{table}[h!]
\begin{center}
\begin{tabular}{lrr}
\toprule
metoda & średnia dokładność & odchylenie std \\
\midrule
--      & 0.7592	& 0.0306 \\
TopDown & --		& --	\\
BottomUp& --		& --	\\
\bottomrule
\end{tabular}
\caption{Dokładność klasyfikacji - klasyfikator Bayesa, 5-krotna walidacja krzyżowa, zadana ilość przedziałów = X}
\label{tab:bayes_td_bu_comp_full_set}
\end{center}
\end{table}



\chapter{Pakiet R}
Projekt został zaimplementowany jako pakiet języka R, \emph{discretize}, posiadający następujący interfejs:
\begin{itemize}
\item{CrossValidateBayes}
\item{RequestedIntervalsNumCriterion}
\item{TopDown}
	\begin{itemize}
	\item{predict}
	\item{print}
	\item{summary}
	\end{itemize}
\item{DeltaCriterion}
\item{MinEntropyDecreaseCriterion}
\item{BottomUp}
	\begin{itemize}
	\item{predict}
	\item{print}
	\item{summary}
	\end{itemize}
\item{MinChiCriterion}
\end{itemize}

\noindent Każda publiczna metoda została udokumentowana (polecenie help()), a do pakietu dołączono testy jednostkowe napisane przy użyciu RUnit, wykonywane podczas budowania pakietu i wywoływane poleceniem "R CMD check katalog\_pakietu". Do pakietu został dołączony zestaw danych batch1.dat~\cite{Gas:2012}, który można załadować przy użyciu wywołania data(gas1). Pakiet został sprawdzony i zbudowany bez ostrzeżeń przy użyciu poleceń:
\begin{alltt}
R CMD check discretize
R CMD build --resave-data discretize
\end{alltt}
\noindent Oraz zainstalowany poleceniem:
\begin{alltt}
R CMD INSTALL --clean discretize_1.0.tar.gz
\end{alltt}


\nocite{*}
\bibliography{main}


\end{document}
