Projekt został zaimplementowany jako pakiet języka R, \emph{discretize}, posiadający następujący interfejs:
\begin{itemize}
\item{CrossValidateBayes}
\item{RequestedIntervalsNumCriterion}
\item{TopDown}
	\begin{itemize}
	\item{predict}
	\item{print}
	\item{summary}
	\end{itemize}
\item{DeltaCriterion}
\item{MinEntropyDecreaseCriterion}
\item{BottomUp}
	\begin{itemize}
	\item{predict}
	\item{print}
	\item{summary}
	\end{itemize}
\item{MinChiCriterion}
\end{itemize}

\noindent Każda publiczna metoda została udokumentowana (polecenie help()), a do pakietu dołączono testy jednostkowe napisane przy użyciu RUnit, wykonywane podczas budowania pakietu i wywoływane poleceniem "R CMD check katalog\_pakietu". Do pakietu został dołączony zestaw danych batch1.dat~\cite{Gas:2012}, który można załadować przy użyciu wywołania data(gas1).
